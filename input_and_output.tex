% !TEX root = python3_cheetsheet.tex

\section{入出力}

\subsection{文字列}

ある文字列Sを読み込む際は\texttt{input()}を用いる。

\begin{lstlisting}[language=Python]
  S = input()
\end{lstlisting}

\subsection{数値(int or float)}

整数及び浮動小数点を読み込みたい場合は\texttt{input()}で文字列として
読み込んだ後、\texttt{int}でキャストしてやればいい。

\begin{lstlisting}[language=Python]
  N = int(input())
\end{lstlisting}

\subsection{配列}
\label{sec:list}

もし、$a_0, a_1, ..., a_n$のような入力を配列として読み込みたい場合は
\texttt{split()}で切り分けて変数に入れることでおk。

\begin{lstlisting}[language=Python]
  a = input().split()
\end{lstlisting}

\newpage

\subsection{数値配列}
\texttt{split()}したlistをすべて数値として入力させたいときは、
\texttt{map(int, list)}とすればすべての要素をint型に変換できる。

\begin{lstlisting}[language=Python]
  a = list(map(int, input().split()))
\end{lstlisting}

floatなど他の型にもできる。

\begin{lstlisting}[language=Python]
  a = list(map(float, input().split()))
\end{lstlisting}

\subsection{一度に複数の変数へ代入}

例えば以下のような入力があったする。

\begin{lstlisting}
  10 5
\end{lstlisting}

それぞれの数値を$m$、$r$という変数に代入したい場合、アンパック代入を用いて
要素ごとに左辺の変数に代入できる。

\begin{lstlisting}[language=Python]
  m, r = list(map(int, input().split()))
\end{lstlisting}
