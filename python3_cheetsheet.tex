\documentclass[uplatex, 11pt]{jsarticle}

\usepackage[dvipdfmx]{graphicx}
\usepackage{float}
\usepackage[dvipdfmx]{hyperref}
\usepackage{pxjahyper}
\usepackage{docmute}
\usepackage{plistings}
\usepackage{color}

\definecolor{pblue}{rgb}{0.13,0.13,1}
\definecolor{pgreen}{rgb}{0,0.5,0}
\definecolor{pred}{rgb}{0.9,0,0}
\definecolor{pgrey}{rgb}{0.46,0.45,0.48}

\lstset{
  frame=single,
  tabsize=2,
  basicstyle=\ttfamily,
  keepspaces=true,
  showstringspaces=false,
  breakatwhitespace=true,
  commentstyle=\color{pgreen},
  keywordstyle=\color{pblue},
  stringstyle=\color{pred},
  basicstyle=\ttfamily,
  moredelim=[il][\textcolor{pgrey}]{\$\$},
  moredelim=[is][\textcolor{pgrey}]{\%\%}{\%\%}
}
\hypersetup{
  pdftitle={ACM ICPC Python3 CheetSheet},
  pdfauthor={Aya Tokikaze}
}

\renewcommand{\headfont}{\bfseries}
\renewcommand\thefootnote{\arabic{footnote}}

\title{ACM ICPC Python3 CheetSheet}
\author{Aya Tokikaze}

\begin{document}
\maketitle

\tableofcontents

% !TEX root = python3_cheetsheet.tex

\section{入出力}

\subsection{文字列}

ある文字列Sを読み込む際は\texttt{input()}を用いる。

\begin{lstlisting}[language=Python]
  S = input()
\end{lstlisting}

\subsection{数値(int or float)}

整数及び浮動小数点を読み込みたい場合は\texttt{input()}で文字列として
読み込んだ後、\texttt{int}でキャストしてやればいい。

\begin{lstlisting}[language=Python]
  N = int(input())
\end{lstlisting}

\subsection{配列}
\label{sec:list}

もし、$a_0, a_1, ..., a_n$のような入力を配列として読み込みたい場合は
\texttt{split()}で切り分けて変数に入れることでおk。

\begin{lstlisting}[language=Python]
  a = input().split()
\end{lstlisting}

\newpage

\subsection{数値配列}
\texttt{split()}したlistをすべて数値として入力させたいときは、
\texttt{map(int, list)}とすればすべての要素をint型に変換できる。

\begin{lstlisting}[language=Python]
  a = list(map(int, input().split()))
\end{lstlisting}

floatなど他の型にもできる。

\begin{lstlisting}[language=Python]
  a = list(map(float, input().split()))
\end{lstlisting}

\subsection{一度に複数の変数へ代入}

例えば以下のような入力があったする。

\begin{lstlisting}
  10 5
\end{lstlisting}

それぞれの数値を$m$、$r$という変数に代入したい場合、アンパック代入を用いて
要素ごとに左辺の変数に代入できる。

\begin{lstlisting}[language=Python]
  m, r = list(map(int, input().split()))
\end{lstlisting}


\section{制御構造}

\subsection{三項演算子}
pythonでの三項演算子は
\begin{lstlisting}[language=Python]
  <条件式がTrueのときの値> if <条件式> else <Falseのときの値>
\end{lstlisting}

でできる。

\begin{lstlisting}[language=Python]
  >>> a = True
  >>> print('wei' if a else 'soiya')
  wei
\end{lstlisting}

\subsection{スワップ}

pythonでのスワップはcみたいに値をtmpに突っ込まなくても、以下のようにすればできる。

\begin{lstlisting}[language=Python]
  a, b = b, a
\end{lstlisting}

% !TEX root = python3_cheetsheet.tex

\section{スライス}
スライスは配列や文字列の任意の部分もしくは範囲を取り出す方法。けっこう便利なので
知っておくべき。

以下のような配列$a$を定義しておく。
\begin{lstlisting}[language=Python]
  a = [0, 1, 2, 3, 4, 5, 6, 7, 8, 9]
\end{lstlisting}

\subsection{$N$番目の要素を取り出す}

基本中の基本、説明することはない。
\begin{lstlisting}[language=Python]
  a[N]

  >>> a[1]
  1
\end{lstlisting}

\subsection{末尾からの取り出し}

負の数を指定すると配列の末尾からの指定になる

\begin{lstlisting}[language=Python]
  # -1で一番最後
  >>> a[-1]
  9

  >>> a[-8:-5]
  [2, 3, 4]

  # 始点省略
  >>> a[:-6]
  [1, 2, 3]

  # 終点省略
  >>> a[-2:]
  [8, 9]
\end{lstlisting}

\newpage

\subsection{$N$番目から$M$番目の要素を取り出す}
コロンで区切ると範囲指定ができる。
\begin{lstlisting}[language=Python]
  # n番目からm-1番目の要素を取り出す
  a[n:m]

  # 1番目から3番目の要素を取り出す
  >>> a[1:4]
  [1, 2, 3]

  # 始点を省略すると0番目から
  >>> a[:3]
  [0, 1, 2, 3]

  # 終点を省略すると最後まで
  >>> a[7:]
  [7, 8, 9]
\end{lstlisting}


\subsection{ステップ数を指定して取り出し}

奇数項だけ取り出したいときとかにつかう

\begin{lstlisting}[language=Python]
  # コロンを2個続けるとステップして抜き出します
  >>> a[::3]
  [1, 4, 7]

  # もちろん切り出しつつのステップも可能
  >>> a[2:7:3]
  [3, 6]

  # 負の数を使って後ろから
  >>> a[::-3]
  [9, 6, 3]

  # ややこしい・・・orz
  >>> a[-2:-8:-3]
  [8, 5]

  # コロンが2個あって且つ後ろからステップする場合は
  # 逆順にしてから抜き出してるっぽい
  # よって以下は何も取り出せない
  >>> a[-8:-2:-3]
  []
\end{lstlisting}

\subsubsection{スライスを用いたリバース}
\texttt{reversed()}よりもこっち使ったほうが簡潔だしいいと思います。

\begin{lstlisting}[language=Python]
  # 配列の末尾から取り出し
  >>> a[::-1]
  [9, 8, 7, 6, 5, 4, 3, 2, 1, 0]
\end{lstlisting}


\section{数値関係}

\subsection{数値の桁数}

\begin{lstlisting}[language=Python]
  >>> a = 1234567
  >>> int(math.log10(a) + 1)
  7
\end{lstlisting}

\section{文字列とか}

\subsection{0で桁あわせ}
0埋めの桁合わせをしたいときは一旦strに変換してから\texttt{zfill}メソッドで指定した
桁数分0埋めできる。
\begin{lstlisting}[language=Python]
  # 元のすうち
  >>> a = 12345
  # 合わせたい桁
  >>> l = 7
  >>> str(a).zfill(l)
  0012345
\end{lstlisting}











\end{document}
